\chapter{Architectural Views} \label{ArchitecturalViews}


\section{Context View}

Refer to \textit{R\&W Chapter 16}. 

\subsection{Stakeholders’ uses of this view}

\subsection{Context Diagram}

Context Diagram should display all external entities that may interact with the system. This section should include a \textbf{Context Diagram and explanations} for the context diagram.

\subsection{External Interfaces}
This section should include an \textbf{External Interfaces Class Diagram}. Descriptions of the operations given in the external interface class diagram should also be given. \textbf{You should aim for 3 external interfaces.}

\subsection{Interaction scenarios}
This section includes \textbf{2 Activity Diagrams} to show interaction sequences taking place over the external interfaces. Choose the 2 most complex interactions for activity diagrams. They must be different from those in your \gls{srs} document.




\section{Functional View}

Refer to \textit{R\&W Chapter 17}. 

\subsection{Stakeholders’ uses of this view}

\subsection{Component Diagram}

This section should include a \textbf{Component Diagram and explanations} for the component diagram. The provides/requires relationships between components must be shown.

\subsection{Internal Interfaces}
This section should include an \textbf{Internal Interfaces Class Diagram}. Descriptions of the operations given in the internal interface class diagram should also be given. \textbf{You should aim for 4 internal interfaces.}

\subsection{Interaction Patterns}
This section includes \textbf{3 Sequence Diagrams} to show messaging sequences taking place among the system components over the internal interfaces. Choose the 3 most complex interactions for sequence diagrams. They must be different from those in your \gls{srs} document.



\section{Information View}

Refer to \textit{R\&W Chapter 18}. 

\subsection{Stakeholders’ uses of this view}

\subsection{Database Class Diagram}

\textbf{Database Class Diagram} involving the key database or main memory objects. Complete with relevant associations. Descriptions of the non-obvious names (for classes, attributes, operations) should also be given.

\subsection{Operations on Data}
Descriptions of the operations are given in the database class diagram. These operations may deal with the storage and handling of information regarding stores, customers, products, and so on. \textbf{Operations should be listed in a table or using bullets.}
These usually include CRUD (Create Read Update Delete) operations.



\section{Deployment View}

Refer to \textit{R\&W Chapter 21}. 

\subsection{Stakeholders’ uses of this view}

\subsection{Deployment Diagram}

This section should include a \textbf{Deployment Diagram and explanations} for the deployment diagram.

\section{Design Rationale}
State \textbf{one rationale} specifically referring to each view presented