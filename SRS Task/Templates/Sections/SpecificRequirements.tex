\chapter{Specific Requirements} \label{specificRequirements}

Refer to \textit{(Clause 9.6.10)}. 

\section{External Interfaces}

\textbf{External Interfaces Class Diagram and its explanations} go here. Plus, other content as appropriate

Refer to \textit{(Clause 9.6.11, 9.5.8)}. 

\section{Functions}

\textbf{Use-case diagram} goes here; \textbf{detailed use-case descriptions in a reasonable template} follow. You are expected to have \textbf{about 10 use cases covering major system functionality}. Have some associations in your use-case diagram, e.g. include, extend, specialization. Choose \textit{three} most complicated use cases. Construct three diagrams \textbf{(one sequence diagram, one activity diagram, and one state diagram)} to elaborate on these three use cases. Plus, other content as appropriate.

Refer to \textit{(Clause 9.6.1)}. 

\section{Logical Database Requirements }
Key data objects (persistent or not) and their major attributes. Draw the \textbf{Class Diagram} with associations. A class dictionary can be omitted, provided that the naming is understandable.

Refer to \textit{(Clause 9.6.15)}.

\section{Design Constraints}
Specify constraints on the system design imposed by external factors, such as official standards, regulatory requirements, or organizational/managerial limitations.\\ Refer to \textit{(Clause 9.6.16)}.

\section{System Quality Attributes}
Important quality attributes (Usability \textit{(Clause 9.6.13, 9.5.6)}, Performance \textit{(Clause 9.6.14, 9.5.7)}, Dependability properties, Maintainability, and so on) in the order of priority with associated requirements.

Refer to \textit{(Clause 9.6.18)}.


\section{Supporting Information}

Refer to \textit{(Clause 9.6.20)}


